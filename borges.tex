\documentclass[11pt]{article}

\usepackage[utf8]{inputenc}
\usepackage[a4paper, left=2cm, right=2cm, top=2cm, bottom=2cm]{geometry}
\usepackage[T1]{fontenc}
\usepackage{amssymb}
\usepackage{amsmath}
\usepackage{amsthm}   %důkazy
\usepackage[czech]{babel}
% To enable encTeX with automatic nonbreakable spaces in CZ/SK text
% http://merlin.fit.vutbr.cz/wiki/index.php/%C4%8Cesk%C3%A1_sazba_v_LaTeXu
%\usepackage{encxvlna}

\setlength{\parskip}{0.5em}

\begin{document}
\noindent
\Huge
\textbf{Babylonská knihovna}\footnote{Povídka z~knihy Tvař a maska (Odeon 1989), překlad Kamil Uhlíř, Josef Forbelský a František Vrhel.}

\Large\noindent
\textit{Jorge Luis Borges}
\bigskip

\normalsize
Vesmír (který jiní nazývají Knihovnou) je vytvářen šestiúhelníkovými galeriemi, jejichž počet je neurčitý a možná i nekonečný. Uprostřed galerií jsou velké větrací šachty, obehnané nizoučkým zábradlím. Z~každé galerie jsou až do nekonečna vidět dolní i horní poschodí. Uspořádání všech galerií je naprosto stejné: Podél stěn, s~výjimkou dvou, stojí dvacet regálů, tj.\ vždy pět dlouhých regálů u~stěny. Regály, jež sahají až ke stropu galerií, jsou jen o~málo vyšší než normální knihovník. V~jedné z~nezastavěných stěn je vstup do úzké chodby, která vede do jiné galerie, shodné s~první galerií a se všemi ostatními galeriemi. Po pravé i po levé straně chodby je maličká místnost. V~jedné je možno spát vstoje, v~druhé lze ukojit potřebu vyměšování. Tamtudy vede točité schodiště, propadající se do bezedné hloubky a ztrácející se v~nedohledné výšce. V~chodbě je zrcadlo, které přesně odráží vnější jevy. Na základě tohoto zrcadla lidé někdy usuzují, že Knihovna není nekonečná (kdyby byla skutečně nekonečná, k~čemu pak to iluzivní zdvojování?). Já raději sním o~tom, že leštěné plochy představují a slibují nekonečno\dots Zdrojem světla jsou kulovité plody, kterým se říká lampy. V~každém šestiúhelníku visí dvě napříč proti sobě. Světlo, které vyzařují, je nedostačující a nepřetržité.

Jako všichni lidé z~Knihovny jsem v~mládí cestoval. Putoval jsem z~místa na místo hledaje knihu, možná katalog katalogů. Teď, když mé oči už skoro nemohou rozluštit, co píšu, připravuji se na smrt několik málo mil od rodného šestiúhelníku. Až zemřu, určitě se najdou milosrdné ruce, které mě hodí přes zábradlí. Bezedný vzduch bude můj hrob. Mé tělo bude dlouho padat, zpráchniví a vítr, vyvolaný nekonečným pádem, je rozptýlí. Tvrdím, že Knihovna je nekonečná. Idealisté dovozují, že šestiúhelníkový tvar sálů je nevyhnutelná podoba absolutního prostoru, nebo alespoň naší prostorové intuice. Podle jejich názoru je trojúhelníkový nebo pětiúhelníkový tvar sálu nepředstavitelný. (Mystikové nás přesvědčují, že v~extázi zří kruhovou místnost a v~ní velkou kruhovou knihu s~nepřetržitým hřbetem, uzavírajícím kruh podél stěny. Jejich svědectví je však podezřelé a jejich slova nejasná. Ta kruhová kniha je Bůh.) Prozatím postačí, když zopakuji klasický výrok: \textit{Knihovna je koule, jejíž přesný střed tvoři kterýkoli šestiúhelník, jehož obvod je nedosažitelný.}

Podél každé ze čtyř stěn šestiúhelníku je pět regálů. V~každém regálu je uloženo dvaatřicet knih jednotného formátu. Každá kniha má čtyři sta deset stran, na každé stránce je čtyřicet řádek, na každé řádce kolem osmdesáti černých písmen. Také na hřbetu každé knihy jsou písmena, jež však neudávají či nenaznačují, o~čem se lze dočíst na jejích stránkách. Vím, že se někdy tato nesouvislost zdála záhadná. Dřív než podám stručné rozluštění záhady (i když toto rozluštění otvírá tragické perspektivy, je to snad nejdůležitější dějinná událost), chtěl bych připomenout některá axiomata.

První axioma: Knihovna existuje \textit{ab aeterno}. Žádný rozumně uvažující duch nemůže pochybovat o~této pravdě, jejímž bezprostředním odleskem je budoucí věčnost světa. Člověk, nedokonalý knihovník, může být dílem náhody nebo zlomyslných demiurgů. Vesmír, elegantně vybavený regály, záhadnými svazky, neúnavnými schodišti pro cestovatele a záchodky pro sedícího knihovníka, může být jen dílem boha. Abychom si uvědomili, jak vzdálené je božské od lidského, stačí srovnat neohrabané a roztřesené klikyháky, které má omylná ruka píše na desky knihy, s~ústrojnými písmeny uvnitř: jsou přesná, jemná, sytě černá a nenapodobitelně souměrná.

Druhé axioma: \textit{Počet pravopisných znaků je dvacet pět.}\footnote{V původním rukopise nejsou cifry ani velká písmena. Pravopisná znaménka jsou omezena na čárku a tečku. Tato dvě znaménka, mezera a dvaadvacet písmen abecedy tvoří oněch dvacet pět dostačujících znaků, o~kterých se zmiňuje neznámý autor. \textit{(Pozn. vydavatele)}} Na základě tohoto důkazu bylo možno před třemi lety formulovat obecnou teorii Knihovny a rozřešit uspokojivě problém, který dosud žádná hypotéza nerozluštila: proč skoro všechny knihy nemají formu a jsou chaotické. Kniha, kterou můj otec spatřil v~šestiúhelníku patnáctistého devadesátého čtvrtého okruhu, se skládala z~písmen M C V, perverzně opakovaných od první do poslední řádky. Jiná kniha (velmi používaná v~této oblasti) je pouhé bludiště písmen, ale na předposlední stránce je napsáno \textit{Ó čase tvé pyramidy}. Jak vidno, člověk se musí prokousat hromadou nesmyslných skřeků, slovního harampádí a nesouvislých žvástů, aby se dostal k~jedné rozumné řádce nebo přímému poznatku. (Je mi známo, že existuje hornatá krajina, kde knihovníci zavrhují pověrčivý a nicotný zvyk hledat v~knihách nějaký smysl. Počínat si takto je podle nich stejné jako hledat smysl ve snech nebo v~zmatených čárách na dlani\dots Připouštějí, že vynálezci písma napodobili pětadvacet přirozených znaků, ale zároveň tvrdí, že použití těch znaků je náhodné a že knihy samy o~sobě nedávají žádný smysl. Tento názor, jak ještě uvidíme, není tak zcela neopodstatněný.)

Dlouho se myslelo, že ty nesrozumitelné knihy jsou psány mrtvými nebo starobylými jazyky. Je pravda, že nejstarší lidé -- první knihovníci -- mluvili jazykem značně odlišným od našeho dnešního jazyka. Je pravda, že několik mil vpravo se mluví dialektem a o~devadesát schodů výš se hovoří nesrozumitelnou řečí. To všechno je sice pravda, ale čtyři sta deset stran, na kterých se stále opakují písmena M C V, nemůže být napsáno žádným jazykem, ať by byl sebeprimitivnější nebo ať by šlo o~sebevětší nářeční odchylku. Někteří lidé naznačovali, že každé písmeno může mít vliv na písmeno následující a že písmena M C V~na třetí řádce 71. stránky nemají stejnou hodnotu, jakou může mít tatáž skupina písmen na jiném místě na jiné stránce, ale tato nejasná teze se neujala. Jiní zas uvažovali o~kryptogramech. Jejich domněnka byla všeobecně přijata, i když ne ve stejném významu, jak ji formulovali její autoři.

Před pěti sty lety objevil představený jednoho výš položeného šestiúhelníku\footnote{Dřív spravoval vždy tři šestiúhelníky jeden člověk. Sebevraždy a plicní choroby však toto úměrné rozdělení zničily. Zůstala mi vzpomínka, plná nevýslovné melancholie: někdy jsem mnoho nocí putoval po chodbách i schodištích, aniž jsem potkal jediného knihovníka.} knihu, která byla stejně zmatená jako druhé knihy, ale měla skoro dvě stránky sourodých řádek. Ukázal svůj objev potulnému luštiteli, který prohlásil, že řádky jsou psány v~portugalštině. Jiní mu zas řekli, že jsou napsány v~jidiš. Než uplynulo sto let, podařilo se onen jazyk určit: byl to samojedsko-litevský dialekt guaraníštiny s~názvuky klasické arabštiny. Byl také rozluštěn smysl textu: šlo o~poučky z~kombinatorní analýzy, osvětlované příklady variací s~nekonečným počtem opakování. Příklady umožnily jednomu geniálnímu knihovníkovi objevit základní zákon Knihovny. Tento myslitel upozornil, že všechny knihy, ať seberozdílnější, jsou tvořeny stejnými prvky: mezerou, tečkou, čárkou a dvaadvaceti písmeny abecedy. Poukazoval taky na skutečnost, kterou potvrdili všichni cestovatelé: \textit{V~celé obrovské Knihovně nejsou ani dvě knihy úplně stejné.} Z~těchto nevyvratitelných premis vyvodil závěr, že Knihovna je završený celek a její regály zahrnují všechny možné kombinace (obrovský, i když nikoli nekonečný počet) pětadvaceti ortografických znaků, tedy všechno, co lze vyjádřit ve všech jazycích. Všechno: velice podrobnou historii budoucnosti, vlastní životopisy archandělů, přesný katalog Knihovny, tisíce a tisíce falešných katalogů, důkaz nesprávnosti těchto katalogů, důkaz nesprávnosti pravého katalogu, gnostické evangelium Basileidovo, komentář k~tomuto evangeliu, komentář ke komentáři k~tomuto evangeliu, pravdivé vylíčení tvé smrti, překlad každé knihy do všech jazyků, interpolace každé knihy do všech knih.

Když se rozhlásilo, že Knihovna obsahuje všechny knihy, vyvolalo to nejprve pocit podivného štěstí. Všem připadalo, jako by byli pány nedotčeného a tajného pokladu. Neexistoval osobní nebo světový problém, jehož výmluvné řešení by se nemohlo najít v~některém šestiúhelníku. Vesmír došel svého opodstatnění. Vesmír náhle uchvátil neomezené rozměry naděje. V~té době se mnoho mluvilo o~Knihách ospravedlnění: knihách apologie a věštby, které navždy ospravedlňovaly činy každého člověka ve vesmíru a chovaly zázračná tajemství spjatá s~jeho budoucností. Tisíce lidí lakotně opustily líbezný rodný šestiúhelník, a hnány marnou snahou najít svou Knihu ospravedlnění, vrhly se po schodech nahoru. Tito poutníci se hádali na úzkých chodbách, pronášeli nesrozumitelné kletby, rdousili se na božských schodištích, házeli do hlubokých tunelů klamné knihy a umírali v~propasti, kam je svrhli lidé z~dalekých krajin. Jiní zešíleli\dots Knihy ospravedlnění existují (sám jsem viděl dvě, týkající se budoucích osob, osob, které snad nejsou pomyslné), ale ti, kdo je hledali, si neuvědomili, že je vlastně nemožné, aby člověk našel právě tu svou knihu nebo nějakou její věrolomnou obměnu.

Tenkrát se také doufalo, že dojde k~objasnění původu Knihovny a původu času, dvou největších záhad lidského rodu. Je pravděpodobně možné podat slovní vysvětlení těchto hlubokých záhad. Pokud by nestačil jazyk filosofů, mnohotvárná Knihovna asi vytvořila neslýchaný jazyk, který je zapotřebí, a vytvořila i jeho slovníky a mluvnice. Je tomu již čtyři sta let, co lidé kvůli této věci obtěžují šestiúhelníky\dots Existují oficiální hledači, \textit{inkvizitoři}. Viděl jsem je, jak vykonávají svou funkci. Vracejí se vždy úplně vysílení. Vyprávějí o~jakémsi schodišti bez stupňů, na kterém se málem zabili. Vyprávějí o~galeriích a schodech s~knihovníkem. Někdy vezmou do ruky nejbližší knihu a začnou v~ní listovat, pátrajíce po hanebných slovech. Zřejmě však nikdo z~nich nečeká, že něco najde.

Nehoráznou naději přirozeně vystřídala nesmírná sklíčenost. Jistota, že některý regál v~některém šestiúhelníku skrývá drahocenné knihy a že ty knihy jsou nedostupné, se zdála nesnesitelná. Jedna rouhačská sekta navrhla, aby se zanechalo hledání a aby všichni lidé tak dlouho míchali písmena a znaky, dokud s~nepravděpodobnou pomocí náhody nevytvoří ony kanonické knihy. Úřady pokládaly za svou povinnost vydat přísná nařízení. Sekta zmizela, ale jako chlapec jsem ještě vídal staré lidi, jak se dlouho zavírají na záchodě a kovovými kotoučky nemohoucně napodobují v~zakázané nádobce božský chaos.

Jiní naopak mysleli, že je v~prvé radě nutno odstranit nepotřebná díla. Násilím pronikali do šestiúhelníků, ukazovali pověřovací listiny, které nebyly vždycky padělané, s~nechutí prolistovali jeden svazek a odsoudili pak celé regály. Nesmyslnou zkázu milionů knih nutno přičíst na vrub jejich hygienické, asketické zuřivosti. Jméno těch lidí je prokleto, ale pokud se naříká nad \uv{poklady} zničenými jejich zběsilostí, přehlížejí se dvě obecně známé skutečnosti. Za prvé: Knihovna je tak obrovská, že jakákoli lidská snaha o~její zmenšení je zcela zanedbatelná. Za druhé: Každý svazek je unikátní, nenahraditelný, ale vždy existuje (protože Knihovna je završený celek) několik set nedokonalých faksimile -- děl lišících se navzájem jen písmenem nebo čárkou. Na rozdíl od všeobecné panujícího mínění se odvažuji vyslovit domněnku, že Očišťovači sice plenili, ale následky jejich počínání byly zveličeny hrůzou, kterou tito fanatici vyvolávali. Hnala je horečná touha zmocnit se knih z~Karmínového šestiúhelníku, knih menšího formátu, než je obvyklý formát, knih všemocných, ilustrovaných, magických.

Další pověra, známá z~té doby, je pověra o~Muži, který četl Knihu. Na některém regálu v~některém šestiúhelníku (uvažovali tenkrát lidé) musí existovat kniha, která je zkratkou a dokonalým výtahem ze \textit{všedi ostatních knih}. Nějaký knihovník tu knihu pročetl a podobá se bohu. V~jazyce té oblasti se dosud zachovaly stopy kultu onoho dávného úředníka. Mnoho poutníků se ho vydalo hledat. Celých sto let putovali nejrůznějšími směry. Jak zjistit, kde leží uctívaný tajný šestiúhelník, uchovávající tu knihu? Kdosi navrhl použít regresivní metody: Určit, kde je kniha A~tak, že se nejprve nahlédne do knihy B, která označí místo, kde knihu A~hledat. Určit, kde je kniha B tak, že se nejprve nahlédne do knihy C, a tak do nekonečna\dots V~podobných dobrodružstvích jsem promrhal a strávil svůj život. Nezdá se mi nepravděpodobné, že v~některém regálu ve vesmíru existuje všeobsáhlá kniha.\footnote{Opakuji: K~existenci nějaké knihy stačí, připustí-li se možnost její existence. Vyloučeno je jen to, co je nemožné. Žádná kniha není zároveň schodiště, i když se nepochybně vyskytují knihy, které o~takové možnosti diskutují, popírají a dokazují ji, a vyskytují se jiné knihy, jejichž struktura odpovídá struktuře schodiště.}

Prosím neznámé bohy, aby umožnili jednomu člověku -- jednomu jedinému člověku třebas i před tisíciletími! -- do té knihy nahlédnout a přečíst ji. Nemůže-li se té cti, té moudrosti a toho štěstí dostat mně, ať se jich dostane druhým lidem. Ať nebe existuje, i když mně bude souzeno peklo. Ať jsem zneuctěn a zničen, ale ať v~jednom okamžiku a v~jedné bytosti Tvá obrovská Knihovna prokáže oprávněnost své existence.

Neznabozi tvrdí, že nesmyslnost je v~Knihovně normálním zjevem, kdežto rozumnost (a dokonce i pouhá skromná spojitost) je zde skoro zázračnou výjimkou. Mluví (vím to) o~\uv{horečné Knihovně, jejíž náhodné svazky, neustále ohrožované nebezpečím, že se promění v~jiné knihy, všechno dotvrzují, popírají a matou jako blouznící božstvo}. Tato slova, nejen odhalující chaos, ale taky jej názorně předvádějící, nadmíru jasně dosvědčují velmi špatný vkus a zoufalou neznalost. Knihovna skutečně zahrnuje všechny slovní struktury, všechny variace dvaceti pěti ortografických znaků, ale není v~ní ani jeden naprostý nesmysl. Je zbytečné poukazovat na to, že nejlepší svazek v~mnoha šestiúhelnících, jež spravuji, se nazývá \textit{Učesaný hrom}, jiný zas \textit{Sádrová křeč} a jiný \textit{Axaxaxas mlö}. Pro tyto na první pohled nesouvislé výroky lze nepochybně nalézt odůvodnění, ať v~podobě kryptogramu nebo alegorie. Je to slovní odůvodnění a Knihovna je již \textit{ex hypothesi} obsahuje. Z~písmen
\begin{center}
\textit{dhcmrlchtdj}
\end{center}
\noindent
nelze vytvořit kombinaci, kterou by Knihovna nebyla předvídala a která by v~některém z~jejích tajných jazyků neměla strašný smysl. Nikdo nemůže vyslovit slabiku, která by nebyla plná něhy a obav, která by v~některém z~těch jazyků nebyla mocným jménem Boha. Mluvit znamená dopouštět se tautologií. Má zbytečná a upovídaná epištola už existuje v~jednom z~třiceti svazků v~pěti regálech jednoho z~nespočetných šestiúhelníků -- a taky tam existuje popření této epištoly. (Počet \textit{n} všech možných jazyků užívá stejného slovníku. V~některých jazycích znak \textit{knihovna} připouští správnou definici: \textit{všudypřítomná a věčná soustava šestiúhelníkových galerií}, ale znak \textit{knihovna} je také \textit{chléb} nebo \textit{pyramida} nebo jakákoli jiná věc, a šest slov, tvořících jeho definici, má jiný smysl. Ty, kdo mě čteš, jsi si jist, že rozumíš mému jazyku?)

Tím, že soustavně píšu, vymaňuji se ze současného stavu lidstva. Jistota, že už bylo všechno napsáno, nás znemožňuje a připodobňuje nás přízrakům. Znám končiny, kde mladí lidé padají na kolena před knihami a barbarsky líbají jejich listy, ale nedovedou rozluštit ani jedno písmeno. Obyvatelstvo je zdecimováno epidemiemi, kacířskými rozbroji i putováním, které se nevyhnutelně zvrhají v~loupežnictví. Zmínil jsem se myslím o~sebevraždách, jež jsou rok co rok početnější. Možná, že mě klame stáří a bázlivost, ale tuším, že lidský rod -- jediný, který existuje -- je na vymření a že jen Knihovna, osvětlená, osamělá, nekonečná, dokonale nehybná, plná drahocenných svazků, zbytečná, nezničitelná, tajná, potrvá navěky.

Právě jsem napsal \textit{nekonečná}. Tohoto přídavného jména jsem nepoužil jen tak z~rétorického návyku. Myšlence, že svět je nekonečný, neschází totiž podle mého názoru logičnost. Ti, kdo pokládají svět za ohraničený, vycházejí z~předpokladu, že na nějakých velmi vzdálených místech chodby schodiště a šestiúhelníky mohou nepochopitelně přestat -- což je absurdní. Ti, kdo si svět představují bez hranic, zapomínají zas, že možný počet knih je ohraničený. Odvažuji se navrhnout takovéto řešení starého problému: \textit{Knihovna je neohraničená a periodická.} Kdyby se věčný cestovatel ubíral Knihovnou kterýmkoliv směrem, zjistil by po uplynutí několika staletí, že tytéž svazky se opakují v~témž chaotickém pořadí (které by se při tomto opakování stalo urovnaným pořadím: stalo by se Řádem). Je to elegantní naděje, jež obveseluje mou samotu.\footnote{Jak upozornila Letizia Álvarezová de Toledo, je obrovská Knihovna zbytečná. Přesně vzato by stačil \textit{jediný svazek} běžného formátu, vytištěný borgisem nebo garmondem. Svazek by měl nekonečný počet stran nekonečně tenkého papíru. (Na počátku XVII. století prohlásil Cavalieri, že každé pevné těleso se skládá z~nekonečného počtu ploch vrstvených na sebe.) Manipulace s~tímto hedvábným \textit{vademecum} by nebyla pohodlná: Každý viditelný list by se rozkládal v~další obdobné listy. Nepředstavitelný prostřední list by neměl druhou stranu.}

\raggedleft
\textit{1941, Mar del Plata}

\end{document}
