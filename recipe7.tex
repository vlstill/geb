\documentclass[12pt]{article}

\usepackage[utf8]{inputenc}
\usepackage[a4paper, left=2.5cm, right=2.5cm, top=2.5cm, bottom=2.5cm]{geometry}
\usepackage[T1]{fontenc}
\usepackage{amssymb}
\usepackage{amsmath}
\usepackage{amsthm}   %důkazy
\usepackage[czech]{babel}
\usepackage{paralist}

\begin{document}

\section*{Recept 7: Jogurtová bábovka}

Ingredience:
\begin{compactitem}
    \item 1 hrnek hladké mouky
    \item 1  hrnek polohrubé mouky
    \item 1/2 prášku do pečiva
    \item 1/2 hrnku oleje
    \item 1/2 hrnku cukru (případně i trochu méně, i z 1/4 hrnku to jde, zvlášť pokud je tam i čokoláda)
    \item 1 bílý jogurt (200 ml)
    \item 1 polévková lžíce kakaa
    \item 1 vejce
    \item rozinky nebo čokoládová rýže, případně podle chuti kousky čokolády, kandovaného ovoce nebo ořechy
        (pokud je kvalitní čokoláda na vaření tak doporučuji tu)
\end{compactitem}

\bigskip\noindent
Standardně se jako hrnek uvažuje objem zhruba 200 ml, nicméně verze na semináři
byla z dvojnásobné dávky.

Všechny ingredience dobře promícháme a pečeme asi 35 minut
(bude-li větší, podle potřeby déle) na cca 180 stupňů (když se do toho píchne špejlí
nemělo by to lepit). Tu dvojnásobnou verzi jsem pekl asi 45 minut. Pokud chceme dělat
dvoubarevnou, pak těsto umícháme bez kakaa, půlku těsta odebereme a smícháme s kakaem
a pak nalijeme obě poloviny do formy aniž by se moc míchaly.
\end{document}
