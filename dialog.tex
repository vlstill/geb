\documentclass[12pt]{article}

\usepackage[utf8]{inputenc}
\usepackage[a4paper, left=2.5cm, right=2.5cm, top=2.5cm, bottom=2.5cm]{geometry}
\usepackage[T1]{fontenc}
\usepackage{amssymb}
\usepackage{amsmath}
\usepackage{amsthm}   %důkazy
\usepackage[czech]{babel}

\begin{document}

\section*{6. Kánon ve zvětšených intervalech}
\begin{description}
\item[A:] Tedy vy to s těmi hůlkami umíte, pane Ž.

\item[Ž:]

\item[A:] Nesmírně! Nikdy před tím jsem v čínské restauraci nejedl, ale
    tahle večeře nebyla pro začátek špatná. Jak jste na tom, máte čas chvilku
    si popovídat a posedět, nebo někam spěcháte?

\item[Ž:]

\item[A:] Možná toho víte více než já o čínské kuchyni, pane Ž,ale vsadím
    se s vámi, že já zase mnohem líp znám Japonskou poezii. Četl jste někdy
    nějaké haiku?

\item[Ž:]
    
\item[A:] Smysl, ten přec tkví\ \ \ v srdci posluchačově,\ \ \ lež i v haiku jest.

\item[Ž:]

\item[A:] Ano, prosím. Hmm, ty sušenky vypadají náramně chutně. \textit{sebrat, žvíkat}
    Počkat!! Co je to tam uvnitř!? Nějaká divná věc! Chutná to jako kus papíru! \textit{wtf}

\item[Ž:]

\item[A:] Je to trochu divné, všechna písmena jsou nalepena k sobě, zcela bez mezer.
    A samozřejmě bez háčků a čárek. Není to nějaká šifra? Á, počkat, mně už to vychází!
    Jen tam jenom potřeba doplnit chybějící mezery. Je tam napsáno \uv{U mě ni oběda}.
    To je nějaký nesmysl, ne? Vy tomu rozumíte? Asi to původně byla nějaká podobná báseň
    jako haiku a já jsem většinu slabik spolkl.

\item[Ž:]

\item[A:] \textit{podat} Ale jistě.

\item[Ž:]

\item[A:] Máte naprostou pravdu. Ta báseň obsahuje referenci sama na sebe, to je prostě
    úžasné! Jedna báseň!

\item[Ž:]

\item[A:] Zajímalo by mě, co je napsané uvnitř té vaší sušenky.

\item[Ž:]

\item[A:] Vidíte, váš osud je taky tvaru haiku. Tedy, rozhodně alespoň má
    17 slabik ve třech řádcích 5-7-5.

\item[Ž:]

\item[A:] Řekl bych, že to krásně ilustruje, jak každý chápeme všeliká poselství
    po svém. \textit{pozorovat čajové lístky v hrnku}

\item[ž:]

\item[A:] Ano, děkuji. Mimochodem, jak se daří vašemu příteli Krabovi? Hodně jsem na něj
    myslel od té doby, co jste mi vyprávěl o vašem prazvláštním souboji s gramofonem.

\item[Ž:]

\item[A:] Povězte mi o tom něco, prosím! Jukeboxy se všemi těmi blikacími barevnými
    žárovičkami a sentimentálními šlágry ve mě vyvolávají krásné pocity nostalgie
    a stesku po starých dobrých časech.

\item[Ž:]

\item[A:] Nedovedu si představit, proč by měl být ten jukebox tak velký.
    Leda, že by obsahoval nějakou závratně rozsáhlou sbírku desek. Je to tak?

\item[Ž:]

\item[A:] Cože? Jukebox s jedinou deskou? To je v příkrém rozporu samo se sebou.
    Tak proč je ten jukebox tak velký? Je snad ta jeho osamělá deska nějak
    nestvůrně rozměrná? Má pět metrů v průměru?

\item[Ž:]

\item[A:] Pane Želvo, vy si ze mě děláte legraci. Kdo kdy viděl hrací
    skříň, co umí jenom jednu písničku.

\item[Ž:]

\item[A:] Všechny hrací skříně, na které jsem doposud natrefil, dodržovali základní
    pravidlo: jedna deska, jedna písnička.

\item[Ž:]

\item[A:] Načež se deska roztočí a hudba se rozezní.

\item[Ž:]

\item[A:] To jsem si skoro mohl myslet. Ale pořád mi nejde do hlavy, jak
    se dá s pomocí takového bláznivého zařízení zahrát na jedné desce víc než
    jedna písnička.

\item[Ž:]

\item[A:] Takže předpokládám, že přehrávač B1 si to přihasil po koleji
    a začal rotovat?

\item[Ž:]

\item[A:] Jak bych na ni mohl zapomenout?

\item[Ž:]

\item[A:] Snad mi nebudete tvrdit, že C3 zahrál jinou píseň.

\item[Ž:]

\item[A:] No jasně, vždyť na tom vlastně nic není. Zahrál skladbu na druhé straně desky.
    Nebo spustil přenosku o drážku dál.

\item[Ž:]

\item[A:] To je hloupost! Z jedné desky přece NEJDE vytáhnout dvě různé skladby!

\item[Ž:]

\item[A:] A jak vlastně zněla ta druhá skladba?

\item[Ž:]

\item[A:] To je ale přece úplně jiná melodie!

\item[Ž:]

\item[A:] A není náhodou John Cage skladatel moderní hudby? Mám dojem, že jsem o něm
    zachytil zmínku v jedné knize o haiku.

\item[Ž:]

\item[A:] Dovedu si představit, že kdybych třeba seděl v hlučné kavárně, s radostí
    bych obětoval bůra a nechal bych si zahrát 4'33'' třikrát za sebou. To byla, pane, úleva!

\item[Ž:]

\item[A:] Rozumím-li správně vaším narážkám, pak naznačujete, že byste Cage nejradši
    strčil do klece. To mi vcelku dává smysl. Zase mi ale není moc jasné, jak je
    to s tou Krabovou hrací skříní. Jek mohou být BACH i CAGE zakódování na stejné
    desce?

\item[Ž:]

\item[A:] Počkejte, to bych mohl zvládnout. Tak jako prví bod programu
    máme pokles o jeden půltón, přesněji z B na A, pak výstup o tři půltóny
    na C, a konečně pokles o jeden půltón z H do C. Získáváme vzorec $-1, +3, -1$.

\item[Ž:]

\item[A:] V tom případě tu máme nejprve pokles o 3 půltóny, pak výstup o
    10 půltónů, skoro celou oktávu, a konečně další pokles o tři půltóny.
    Tentokrát vzorec vypadá jako $-3, +10, -3$. To je dost podobné
    tomu předchozímu schématu, nezdá se vám?

\item[Ž:]

\item[A:] No to mě podržte a zase pusťte! Znamená to, že v drážkách té desky
    je pouze zakódováno nějaké rámcové schéma a různé přehrávače si ho interpretují
    po svém?

\item[Ž:]

\item[A:] Nenapínejte mě!

\item[Ž:]

\item[A:] Má tohle násobení intervalů nějaký název?

\item[Ž:]

\item[A:] Úžasné! Takže tři různé melodie, které jste slyšel, byly jen rozličnými
    výsledky zvětšování intervalů jednoho jediného záznamu ukrytého v drážkách té desky.

\item[Ž:]

\item[A:] Připadá mi pozoruhodné, že zvětšením BACH dostanete CAGE a jeho dalším
    zvětšením dostanete zase zpátky BACH, i když trochu přerovnané.
    Budí to dojem, jako by BACH při průchodu mezifází CAGE dostal prudkou žaludeční
    nevolnost.

\item[Ž:]

\end{description}

\section*{7. }
\end{document}
