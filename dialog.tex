\documentclass[12pt]{article}

\usepackage[utf8]{inputenc}
\usepackage[a4paper, left=2.5cm, right=2.5cm, top=2.5cm, bottom=2.5cm]{geometry}
\usepackage[T1]{fontenc}
\usepackage{amssymb}
\usepackage{amsmath}
\usepackage{amsthm}   %důkazy
\usepackage[czech]{babel}

\begin{document}

\section*{6. Kánon ve zvětšených intervalech}
\begin{description}
\item[A:] Tedy vy to s těmi hůlkami umíte, pane Ž.

\item[Ž:]

\item[A:] Nesmírně! Nikdy před tím jsem v čínské restauraci nejedl, ale
    tahle večeře nebyla pro začátek špatná. Jak jste na tom, máte čas chvilku
    si popovídat a posedět, nebo někam spěcháte?

\item[Ž:]

\item[A:] Možná toho víte více než já o čínské kuchyni, pane Ž,ale vsadím
    se s vámi, že já zase mnohem líp znám Japonskou poezii. Četl jste někdy
    nějaké haiku?

\item[Ž:]
    
\item[A:] Smysl, ten přec tkví\ \ \ v srdci posluchačově,\ \ \ lež i v haiku jest.

\item[Ž:]

\item[A:] Ano, prosím. Hmm, ty sušenky vypadají náramně chutně. \textit{sebrat, žvíkat}
    Počkat!! Co je to tam uvnitř!? Nějaká divná věc! Chutná to jako kus papíru! \textit{wtf}

\item[Ž:]

\item[A:] Je to trochu divné, všechna písmena jsou nalepena k sobě, zcela bez mezer.
    A samozřejmě bez háčků a čárek. Není to nějaká šifra? Á, počkat, mně už to vychází!
    Jen tam jenom potřeba doplnit chybějící mezery. Je tam napsáno \uv{U mě ni oběda}.
    To je nějaký nesmysl, ne? Vy tomu rozumíte? Asi to původně byla nějaká podobná báseň
    jako haiku a já jsem většinu slabik spolkl.

\item[Ž:]

\item[A:] \textit{podat} Ale jistě.

\item[Ž:]

\item[A:] Máte naprostou pravdu. Ta báseň obsahuje referenci sama na sebe, to je prostě
    úžasné! Jedna báseň!

\item[Ž:]

\item[A:] Zajímalo by mě, co je napsané uvnitř té vaší sušenky.

\item[Ž:]

\item[A:] Vidíte, váš osud je taky tvaru haiku. Tedy, rozhodně alespoň má
    17 slabik ve třech řádcích 5-7-5.

\item[Ž:]

\item[A:] Řekl bych, že to krásně ilustruje, jak každý chápeme všeliká poselství
    po svém. \textit{pozorovat čajové lístky v hrnku}

\item[ž:]

\item[A:] Ano, děkuji. Mimochodem, jak se daří vašemu příteli Krabovi? Hodně jsem na něj
    myslel od té doby, co jste mi vyprávěl o vašem prazvláštním souboji s gramofonem.

\item[Ž:]

\item[A:] Povězte mi o tom něco, prosím! Jukeboxy se všemi těmi blikacími barevnými
    žárovičkami a sentimentálními šlágry ve mě vyvolávají krásné pocity nostalgie
    a stesku po starých dobrých časech.

\item[Ž:]

\item[A:] Nedovedu si představit, proč by měl být ten jukebox tak velký.
    Leda, že by obsahoval nějakou závratně rozsáhlou sbírku desek. Je to tak?

\item[Ž:]

\item[A:] Cože? Jukebox s jedinou deskou? To je v příkrém rozporu samo se sebou.
    Tak proč je ten jukebox tak velký? Je snad ta jeho osamělá deska nějak
    nestvůrně rozměrná? Má pět metrů v průměru?

\item[Ž:]

\item[A:] Pane Želvo, vy si ze mě děláte legraci. Kdo kdy viděl hrací
    skříň, co umí jenom jednu písničku.

\item[Ž:]

\item[A:] Všechny hrací skříně, na které jsem doposud natrefil, dodržovali základní
    pravidlo: jedna deska, jedna písnička.

\item[Ž:]

\item[A:] Načež se deska roztočí a hudba se rozezní.

\item[Ž:]

\item[A:] To jsem si skoro mohl myslet. Ale pořád mi nejde do hlavy, jak
    se dá s pomocí takového bláznivého zařízení zahrát na jedné desce víc než
    jedna písnička.

\item[Ž:]

\item[A:] Takže předpokládám, že přehrávač B1 si to přihasil po koleji
    a začal rotovat?

\item[Ž:]

\item[A:] Jak bych na ni mohl zapomenout?

\item[Ž:]

\item[A:] Snad mi nebudete tvrdit, že C3 zahrál jinou píseň.

\item[Ž:]

\item[A:] No jasně, vždyť na tom vlastně nic není. Zahrál skladbu na druhé straně desky.
    Nebo spustil přenosku o drážku dál.

\item[Ž:]

\item[A:] To je hloupost! Z jedné desky přece NEJDE vytáhnout dvě různé skladby!

\item[Ž:]

\item[A:] A jak vlastně zněla ta druhá skladba?

\item[Ž:]

\item[A:] To je ale přece úplně jiná melodie!

\item[Ž:]

\item[A:] A není náhodou John Cage skladatel moderní hudby? Mám dojem, že jsem o něm
    zachytil zmínku v jedné knize o haiku.

\item[Ž:]

\item[A:] Dovedu si představit, že kdybych třeba seděl v hlučné kavárně, s radostí
    bych obětoval bůra a nechal bych si zahrát 4'33'' třikrát za sebou. To byla, pane, úleva!

\item[Ž:]

\item[A:] Rozumím-li správně vaším narážkám, pak naznačujete, že byste Cage nejradši
    strčil do klece. To mi vcelku dává smysl. Zase mi ale není moc jasné, jak je
    to s tou Krabovou hrací skříní. Jek mohou být BACH i CAGE zakódování na stejné
    desce?

\item[Ž:]

\item[A:] Počkejte, to bych mohl zvládnout. Tak jako prví bod programu
    máme pokles o jeden půltón, přesněji z B na A, pak výstup o tři půltóny
    na C, a konečně pokles o jeden půltón z H do C. Získáváme vzorec $-1, +3, -1$.

\item[Ž:]

\item[A:] V tom případě tu máme nejprve pokles o 3 půltóny, pak výstup o
    10 půltónů, skoro celou oktávu, a konečně další pokles o tři půltóny.
    Tentokrát vzorec vypadá jako $-3, +10, -3$. To je dost podobné
    tomu předchozímu schématu, nezdá se vám?

\item[Ž:]

\item[A:] No to mě podržte a zase pusťte! Znamená to, že v drážkách té desky
    je pouze zakódováno nějaké rámcové schéma a různé přehrávače si ho interpretují
    po svém?

\item[Ž:]

\item[A:] Nenapínejte mě!

\item[Ž:]

\item[A:] Má tohle násobení intervalů nějaký název?

\item[Ž:]

\item[A:] Úžasné! Takže tři různé melodie, které jste slyšel, byly jen rozličnými
    výsledky zvětšování intervalů jednoho jediného záznamu ukrytého v drážkách té desky.

\item[Ž:]

\item[A:] Připadá mi pozoruhodné, že zvětšením BACH dostanete CAGE a jeho dalším
    zvětšením dostanete zase zpátky BACH, i když trochu přerovnané.
    Budí to dojem, jako by BACH při průchodu mezifází CAGE dostal prudkou žaludeční
    nevolnost.

\item[Ž:]

\end{description}

\newpage
\section*{7. Chromatická fantazie a spor}

\begin{description}

\item[A:] No není to zvláštní? Já jsem na vás také zrovna myslel, jak jsem si
    to rázoval lukami. V tuto dobu jsou tak zelené…

\item[Ž:]

\item[A:] S největší radostí. Tedy, s největší radostí, ale pouze za předpokladu,
    že mě nebude chtít zase vtáhnout do některé z těch vašich proslulých
    záludných logických léček, pane Ž.

\item[Ž:]

\item[A:] Vaše slova mě v skutku uchlácholila, pane Ž., a navíc jste ještě opravdu
    vydráždil mou zvědavost. Opravdu rád si poslechnu, co máte na srdci i na krunýři,
    i když to nebude okázalé.

\item[Ž:]

\item[A:] Že je neobvykle čistý!

\item[Ž:]

\item[A:] Ano, je to nádherný zdravě vyhlížející zelený krunýř. Je příjemné
    na něj popatřit, an se leskne na slunci.

\item[Ž:]

\item[A:] Přece vy sám jste mi právě sdělil, že váš krunýř je zelený.

\item[Ž:]

\item[A:] Pak se shodujeme: je zelený.

\item[Ž:]

\item[A:] Já tu vaši hru asi chápu. Snažíte se naznačit, že to, co vyslovíte,
    není ve skutečnosti pravda, že si želvy hrají s jazykem, že vaše tvrzení neodpovídají
    realitě, že…


    \bigskip
\textit{ž. mu skočí do řeči}
\item[Ž:]

\item[A::] Tak proč zároveň tvrdíte, že váš krunýř je i není zelený?

\item[Ž:]

\item[A:] Byl byste rád, kdybyste to byl řekl?

\item[Ž:]

\item[A:] Ale to je zcela ve sporu s tím, co jste říkal předtím.

\item[Ž:]

\item[A:] Tak tentokrát jsem vás nachytal, kluzký plaze! Načapal jsem vás na švestkách
    a ještě k tomu po uši ve sporu.

\item[Ž:]

\item[A:] A zase! Zabředáváte do sporu hlouběji a hlouběji! Zamotal jste  se do toho
    tak, že se s vámi už nedá vůbec mluvit!

\item[Ž:]

\item[A:] No to už přestává všechno! Taková nehoráznost! Hned vám dokážu, že
    sporný jste vy, a že o tom není sporu!

\item[Ž:]

\item[A:] Hmm… No dobře. Teď tedy zrovna úplně přesně nevím, kde začít.
    Aha…, už vím. Vy jste řekl, že zaprvé váš krunýř je zelený, načež jste jedním
    dechem prohlásil, že zahruhé váš krunýř není zelený. Co bych k tomu měl
    ještě dodávat?

\item[Ž:]

\item[A:] Ale,… to přece, … sakra… Aha, začíná mi to být jasné. (Já jsem někdy takový
    pitomec!) Problém je v tom, že vy a já máme rozdílné představy o tom, co je to spor.
    To je to naše jádro pudla. Tak se podívejte, řeknu vám to, doufám, dostatečně
    srozumitelně: spor nastane, jestliže nějaká osoba vysloví nějaký výrok a současně
    jej sama popře.

\item[Ž:]

\item[A:] Ale já jsem to nemyslel takhle, chtěl jsem říci, že třeba někdo
    něco vysloví a ihned to popře ještě v té samé větě! Nemusí to být úplně ve stejné
    chvíli.

\item[Ž:]

\item[A:] No ano -- dvě věty, které se navzájem vylučují.

\item[Ž:]

\item[A:] Ta vaše úhybná taktika mě někdy mate natolik, že pak už ani nepoznám,
    jestli se dohadujeme o něčem zcela bezvýznamném či o nějakém hlubokém a zcela
    zásadním problému všehomíra.

\item[Ž:]

\item[A:] Jsem naprosto ujištěn, díky. Teď jsem měl možnost se na chvilku
    zamyslet, a myslím, že jsem přišel na nezbytný logický krok, jímž
    vás přesvědčím o tom, že jste sám se sebou ve sporu.

\item[Ž:]

\item[A:] To si pište. S tím budete muset souhlasit dokonce i vy. Myšlenka
    je založená na tom, že protože věříte v platnost výroku 1: \uv{Můj krunýř
    je zelený} A ZÁROVEŇ věříte v platnost výroku 2: \uv{Můj krunýř není zelený},
    plyne z toho, že také věříte ve výrok, který vznikne sloučením obou těchto
    výroků do jedné věty, nebo snad ne?

\item[Ž:]

\item[A:] Výborně, a mám vás! Sloučený výrok, který jsem vám předložil,
    je přece…

\item[Ž:]

\item[A:] Ovšemže. Kdo by si dovolil o tom pochybovat?

\item[Ž:]

\item[A:] No samozřejmě. Copak jsem někdo, kdo neumí abecedu?

\item[Ž:]

\item[A:] Jak není!? Jasně, že je! Je to skvělý slovník, jsou v něm všechny
    důležité biblické postavy!

\item[Ž:]

\item[A:] Jenomže vy ty výroku kombinujete takovým nějakým… nesmyslným způsobem!

\item[Ž:]

\item[A:] No jasně! Vynechal jste mezery! To se nesmí! Místo toho jste měl ty
    výroky pouze sloučit pomocí logické spojky \uv{A}!

\item[Ž:]

\item[A:] Ale ne, je to přece LOGICKÉ! S mojí osobou to nemá co dělat.

\item[Ž:]

\item[A:] Ach pane Ž., ušetřete mě těchto strašlivých muk. Moc dobře víte, co
    znamená logická spojka \uv{A}! Sloučit dvě planá tvrzení pomocí logické spojky
    \uv{A} je přece zcela neškodné.

\item[Ž:] 

\item[A:] Tak teď doopravdy nevím, co mám na to říct. Jsem vaší zásluhou
    zcela bezradný. Cítím se jako ničema, ačkoli mé úmysly byly zcela čisté
    a nevinné.

\item[Ž:]

\item[A:] Je mi hanba, že ve snaze vás přechytračit jsem použil slov, s jejichž
    pomocí jsem vás chtěl vlákat do léčky vnitřního rozporu. Cítím se jako mizera.

\item[Ž:]

\item[A:] Je mi moc líto, že jsem tuto debatu vyprovokoval.

\item[Ž:]

\item[A:] Ano, jistě… Obávám se, že jsem otrokem svých vlastních zlozvyků a že
    se budu na své trnité cestě za pravdou opětovně mýlit a mýlit.

\item[Ž:]

\item[A:] I vám, pane Ž.
\end{description}
\end{document}
