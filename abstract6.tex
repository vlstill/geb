\documentclass[12pt]{article}

\usepackage[utf8]{inputenc}
\usepackage[a4paper, left=2.5cm, right=2.5cm, top=2.5cm, bottom=2.5cm]{geometry}
\usepackage[T1]{fontenc}
\usepackage{amssymb}
\usepackage{amsmath}
\usepackage{amsthm}   %důkazy
\usepackage[czech]{babel}

\begin{document}

\section*{Dialóg 6: Kánon v zväčšených intervaloch}

Achilles a pán Korytnačka sa snažia vyriešiť otázku \uv{Čo obsahuje viac informácie: gramofónová platňa, alebo gramofón, ktorý ju prehráva?} Táto zvláštna otázka vyvstane v okamihu, keď Korytnačka popisuje istú platňu, ktorá pri prehrávaní na rôznych gramofónoch vydáva dve rôzne melódie, a síce B-A-C-H a C-A-G-E. Nakoniec sa však ukáže, že obidve melódie sú v istom zmysle \uv{rovnaké}.

\setcounter{section}{6}
\section*{Kapitola 6: Umiestnenie významu}

Všeobecná diskusia o tom, akým spôsobom je význam rozdelený medzi zašifrovanú správu, dekóder a prijímač. Ilustračné príklady zahŕňajú šroubovnicu DNA, nerozlúštené nápisy dávnych civilizácií a gramofónové záznamy putujúce vesmírnym priestorom. Postuluje sa vzťah medzi inteligenciou a \uv{absolútnym} významom.

\subsection{Význam správy}

Je význam pevnou súčasťou správy, alebo vzniká v interakcii inteligentnej mysle so správou? Bežná gramofónová platňa na prvý pohľad obsahuje význam priamo v sebe -- je v nej vylisovaná konkrétna skladba. Čo však platňa z dialógu? Melódia závisí na použitom prehrávači. Je teda význam ukrytý v platni, v prehrávači alebo snáď rozdelený medzi obidva objekty?

Autor tvrdí, že niektoré správy môžu mať univerzálny, objektívny, význam, ale neplatí to o všetkých správach.

\subsection{Tri vrstvy správy}

Každá správa má (minimálne) 3 vrstvy:
\begin{enumerate}
\item \textbf{Rámcová správa} -- \uv{Som správa, prečítaj ma!}\\ Pochopiť rámcovú správu znamená rozpoznať potrebu dešifrovacieho algoritmu.
\item \textbf{Vonkajšia správa} -- \uv{Som text v slovenčine, zapísaný latinkou.}\\Pochopiť vonkajšiu správu znamená zostrojiť (alebo vedieť zostrojiť) správny dešifrovací algoritmus pre vnútornú správu.
\item \textbf{Vnútorná správa} -- \uv{Koláč je otrávený!}\\Pochopiť vnútornú správu znamená vyťažiť význam, ktorý do nej vložil jej odosielateľ.
\end{enumerate}

\subsection{Genotypy a fenotypy}

\begin{description}
\item[Genotyp] -- nosič informácie, na získanie samotnej informácie však potrebujeme príslušný dekodér. Genotypom je napríklad šroubovnica DNA alebo gramofónová platňa.
\item[Fenotyp] -- samotná informácia, nejakým spôsobom kódovaná genotypom. Napríklad samotný človek alebo hudba.
\end{description}

\noindent
Medzi genotypmi a fenotypmi existuje izomorfizmus:
\begin{align*}
\text{šroubovnica DNA} &\leftrightarrow \text{človek}\\
\text{platňa} &\leftrightarrow \text{hudba}
\end{align*}

\subsection{Spúšťače}

Spúšťače nie sú sami nositeľmi informácie, na získanie významu vyžadujú kontext. Ideálnym príkladom je tlačidlo na jukeboxe -- je len spúšťačom pre danú pieseň v konkrétom kontexte, nie jej genotypom. Autor sa zamýšľa, či náhodou nie je šroubovnica DNA len extrémne komlikováným spúšťačom, skôr ako genotypom. Dokážeme z nej vytvoriť zodpovedajúci fenotyp bez \uv{bežného} chemického kontextu?

\subsection{Myšlienkový experiment}

Predstavme si, že pošleme do vesmíru 4 veci: Gramofónovú platňu s Bachovou sonátou, gramofónovú platňu so skladbou \textit{Imaginárna krajina č.\ 4}\footnote{Toto dielo je klasickou ukážkou \uv{náhodnej hudby}. Skladba vznikla tak, že dvanásť \uv{hudobníkov} náhodne otáčalo spínačmi na dvanástich rádiách.}, jednu molekulu ľudskej DNA a zápis daného genómu pomocou písem AGTC na \uv{kuse} papiera. Nejaká inteligentná rasa tieto objekty zachytí a bude sa snažiť odhaliť ich zmysel.

\begin{description}
\item[Bachova sonáta] Rámcová správa je naznačená neobvyklým tvarom objektu (placka s niekoľkostometrovou drážkou). Rozlúsknutie vonkajšej správy je už zložitejšie (je potrebné postaviť funkčný gramofón). Ak by však aj vyprodukovali totožný zvuk, je ten skutočnou vnútornou správou? Nie sú vnútornou správou emócie, ktoré tá hudba vyvoláva? Má daná rasa vôbec ekvivalent daných emócií?
\item[Imaginárna krajina č.\ 4] S rámcovou a vonkajšou správou je to podobné, ako v predchádzajúcom prípade. Bude však rasa schopná pochopiť vnútornú správu (nezávislosť/vzdor voči existujúcej hudobnej produkcii, snahu \uv{nechať zvuk byť sám sebou},~\dots) bez toho, aby poznali kompletný kontext hudobnej produkcie Zeme za posledných niekoľko storočí?
\item[Šroubovnica DNA] Pochopenie rámcovej správy si asi vieme predstaviť -- ide o pomerne zložitú a istým spôsobom \uv{podozrivo} pravidelnú molekulu. Pri troche predstavivosti, si snáď dokážeme predstaviť, že mimozemská rasa z podoby molekuly dokáže odvodiť konkrétny chemický kontext potrebný na vytvorenie príslušného fenotypu (človeka). Pomáha im skutočnosť, že okrem poradia aminokyselín modeluka DNA nesie informáciu aj v tom, ako reaguje v ktorom chemickom prostredí, z čoho je zložená, \dots
\item[Zápis genómu] Tento prípad sa zdá podobný tomu predošlému -- prenášam plnú genetickú informáciu človeka. Avšak tentoraz boli odstránené všetky dodatočné informácie, ktoré fungovali ako spúšťače na úrovni vonkajšej správy. Mimozemská rasa nevie nič o potrebnom dekodéri.
\end{description}

\noindent
Autor pripisuje \uv{absolútny} (objektívny) význam správam, ktoré nevyžadujú na porozumenie žiadny kontext. Význam takýchto správ sa dá získať len za pomoci inteligencie, bez akýchkoľvek dodatočných znalostí. \textit{Meaning is part of an object to the extent that it acts upon intelligence in a predictable way.}

\end{document}
