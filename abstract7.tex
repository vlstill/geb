\documentclass[12pt]{article}

\usepackage[utf8]{inputenc}
\usepackage[a4paper, left=2.5cm, right=2.5cm, top=2.5cm, bottom=2.5cm]{geometry}
\usepackage[T1]{fontenc}
\usepackage{amssymb}
\usepackage{amsmath}
\usepackage{amsthm}   %důkazy
\usepackage[czech]{babel}
\usepackage{IEEEtrantools}
\usepackage{paralist}
\newcommand{\vnot}[1]{\ensuremath{{\sim}#1}}

\begin{document}

\setcounter{section}{6}
\section{Chromatická fantazie a spor}

\subsection{Dialog}
Achilles se snaží přistihnout pana Želvu při sporu avšak naráží na problém,
když se mu snaží vysvětlit jak má zacházet s výroky.

\subsection{Výroková logika (výrokový počet)}

\subsubsection{Abeceda}
\begin{IEEEeqnarray*}{lClClCl}
    &&            \langle &\quad& \rangle \\
    P     &\quad&   Q     &\quad& R           &\quad& ' \\
    \land &&        \lor  &&      \rightarrow &&        \sim \\
          && [            && ]
\end{IEEEeqnarray*}

\subsubsection{Pravidla vytváření řetězců}
Jestliže $x$ a $y$ jsou dobře utvořené řetězce, pak následující čtyři
řetězce jsou také dobře utvořené:
\begin{compactenum}
\item $\vnot{x}$
\item $\langle x \land y \rangle$
\item $\langle x \lor y \rangle$
\item $\langle x \rightarrow y \rangle$
\end{compactenum}

\subsubsection{Odvozovací pravidla}
\begin{description}
    \item[Pravidlo spojení] Jestliže $x$ a $y$ jsou teorémy, pak řetězec
        $\langle x \land y \rangle$ je také teorém.
    
    \item[Pravidlo oddělení] Jestliže $\langle x \land y \rangle$ je teorém,
        pak řetězce $x$ a $y$ jsou také teorémy.

    \item[Pravidlo dvojí vlnovky] Z jakéhokoliv teorému můžeme odstranit
        řetězec \uv{$\sim\sim$}. Do jakéhokoli teorému ůžeme vložit řetězec \uv{$\sim\sim$} za
        předpokladu, že nově vzniklý řetězec je dobře utvořený. V obou případech je
        výsledný řetězec také teorém.

    \item[Pravidlo fantazie] Jestliže za předpokladu, že $x$ je teorém, můžeme odvodit
        $y$, pak $\langle x \rightarrow y \rangle$ je také teorém ($\equiv$ věta o dedukci ??).

    \item[Pravidlo přenosu] Jsme-li uvnitř nějaké fantazie, můžeme do ní přenést
        a v ní využívat jakýkoliv teorém z \uv{reality} o jednu úroveň výše.

    \item[Pravidlo odloučení] Jestliže $x$ a $\langle x \rightarrow y \rangle$ jsou
        teorémy, pak $y$ je také teorém ($\equiv$ Modus Ponens).

    \item[Pravidlo transpozice (obměny)] Výrazy $\langle x \rightarrow y \rangle$
        a $\langle \vnot{x} \rightarrow \vnot{y} \rangle$ jsou zaměnitelné (ekvivalentní).

    \item[De Morganovo pravidlo] Výrazy $\langle \vnot{x} \land \vnot{y} \rangle$ a
        $\vnot{\langle x \lor y \rangle}$ jsou zaměnitelné.

    \item[Pravidlo prohození] Výrazy $\langle x \land y \rangle$ a
        $\langle \vnot{x} \rightarrow y \rangle$ jsou zaměnitelné.
\end{description}

\end{document}
