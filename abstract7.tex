\documentclass[12pt]{article}

\usepackage[utf8]{inputenc}
\usepackage[a4paper, left=2.5cm, right=2.5cm, top=2.5cm, bottom=2.5cm]{geometry}
\usepackage[T1]{fontenc}
\usepackage{amssymb}
\usepackage{amsmath}
\usepackage{amsthm}   %důkazy
\usepackage[czech]{babel}
\usepackage{IEEEtrantools}
\usepackage{paralist}
\newcommand{\vnot}[1]{\ensuremath{{\sim}#1}}

\begin{document}

\section*{Dialog 7}
Krátký dialog, který kromě názvu nemá s Bachovou \textit{Chromatickou fantazií
a fugou d moll} skoro nic společného. Týká se správného způsobu, jak manipulovat s větami, aby
se zachovala pravdivost výroků, a zejména otázky, zda existují pravidla správného
užívání slova \uv{a}. Tento rozhovor těsně souvisí s Carrolovým dialogem\footnote{
Kapitola 2, dialog Dvojhlasá invence}.

\setcounter{section}{6}
\section{Chromatická fantazie a spor}

Navrhuje způsob, jak pomocí formálních pravidel popsat chování
běžných slov, jako je třeba spojka \uv{a}. Znovu nastoluje myšlenku
izomorfismu a automatického nabývání významu u symbolů ve formálním systému.
Všechny příklady v této kapitole mají formu tvrzenzení, tedy
tvrzení převzatých ze zenových koánů.

\subsection{Výroková logika (výrokový počet)}

\subsubsection{Abeceda}
\begin{IEEEeqnarray*}{lClClCl}
    &&            \langle &\quad& \rangle \\
    P     &\quad&   Q     &\quad& R           &\quad& ' \\
    \land &&        \lor  &&      \rightarrow &&        \sim \\
          && [            && ]
\end{IEEEeqnarray*}

\subsubsection{Pravidla vytváření řetězců}
\textit{Atomy}: $P, Q, R$ jsou atomy. Další atomy můžeme vytvářet tak, že k atomu
přidáme zprava čárku: $R', Q''$.

\medskip\noindent
\textit{Rekurzivní pravidla:} Jestliže $x$ a $y$ jsou dobře utvořené řetězce,
pak následující čtyři řetězce jsou také dobře utvořené:
\begin{compactenum}
\item $\vnot{x}$
\item $\langle x \land y \rangle$
\item $\langle x \lor y \rangle$
\item $\langle x \rightarrow y \rangle$
\end{compactenum}

\subsubsection{Odvozovací pravidla}
\begin{description}
    \item[Pravidlo spojení] Jestliže $x$ a $y$ jsou teorémy, pak řetězec
        $\langle x \land y \rangle$ je také teorém.
    
    \item[Pravidlo oddělení] Jestliže $\langle x \land y \rangle$ je teorém,
        pak řetězce $x$ a $y$ jsou také teorémy.

    \item[Pravidlo dvojí vlnovky] Z jakéhokoliv teorému můžeme odstranit
        řetězec \uv{$\sim\sim$}. Do jakéhokoli teorému ůžeme vložit řetězec \uv{$\sim\sim$} za
        předpokladu, že nově vzniklý řetězec je dobře utvořený. V obou případech je
        výsledný řetězec také teorém.

    \item[Pravidlo fantazie] Jestliže za předpokladu, že $x$ je teorém, můžeme odvodit
        $y$, pak $\langle x \rightarrow y \rangle$ je také teorém ($\equiv$ věta o dedukci).

    \item[Pravidlo přenosu] Jsme-li uvnitř nějaké fantazie, můžeme do ní přenést
        a v ní využívat jakýkoliv teorém z \uv{reality} o jednu úroveň výše.

    \item[Pravidlo odloučení] Jestliže $x$ a $\langle x \rightarrow y \rangle$ jsou
        teorémy, pak $y$ je také teorém ($\equiv$ Modus Ponens).

    \item[Pravidlo transpozice (obměny)] Výrazy $\langle x \rightarrow y \rangle$
        a $\langle \vnot{x} \rightarrow \vnot{y} \rangle$ jsou zaměnitelné (ekvivalentní).

    \item[De Morganovo pravidlo] Výrazy $\langle \vnot{x} \land \vnot{y} \rangle$ a
        $\vnot{\langle x \lor y \rangle}$ jsou zaměnitelné.

    \item[Pravidlo prohození] Výrazy $\langle x \lor y \rangle$ a
        $\langle \vnot{x} \rightarrow y \rangle$ jsou zaměnitelné.
\end{description}

\subsubsection{Zamyšlení nad bezesporností a důkazy obecně}

Dialog mezi paní Obezřetnou a panem Lehkovážným se dotýká problému
dokazatelnosti bezespornosti. Autor se dále zabývá tím, zda je vůbec
možné vytvořit přesvědčivý důkaz a namítá, že důkaz samotný
je v principu závislý na schopnosti čtenáře jej interpretovat,
a čtenář, kterého důkaz nepřesvědčí může požadovat důkaz důkazu,…

\subsection{Zkratky a axiomatizace vyšších úrovní}

Autor navrhuje, že celou myšlenkou typografického odvozovacího systému
je, že nemusíme nad odvozováním přemýšlet, lze ho dělat mechanicky, zatímco
na úrovní přirozeného jazyka je nutné použít inteligenční mód.

Pokud však zavedeme nějakou zkratku, například
$$\langle \vnot{x} \lor \vnot{y} \rangle \equiv \vnot{\langle x \land y \rangle}$$
a dokážeme ji na metaúrovni, pak jsme porušili tuto příjemnou
vlastnost typografického systému a zničili tak celý jeho smysl.

\bigskip
Dále autor uvažuje nad řešením, které spočívá ve formalizaci metateorie
k výrokové logice, což by nám umožnilo mechanicky dokazovat metatvrzení,
avšak jen bychom tím odsunuli problém dále a narazili na něj při úvahách
o metateroii a z toho vyplývající potřebě formalizace metametateorie.

\end{document}
