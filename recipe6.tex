\documentclass[12pt]{article}

\usepackage[utf8]{inputenc}
\usepackage[a4paper, left=2.5cm, right=2.5cm, top=2.5cm, bottom=2.5cm]{geometry}
\usepackage[T1]{fontenc}
\usepackage{amssymb}
\usepackage{amsmath}
\usepackage{amsthm}   %důkazy
\usepackage[czech]{babel}
\usepackage{enumitem}

\begin{document}

\section*{Recept 6: Nugátová platňa z Krabovho jukeboxu}
\bigskip
\begin{minipage}[t]{0.49\textwidth}
\subsection*{Cesto:}
\begin{itemize}[itemsep=0pt]
\item 110\,g hladkej múky
\item 50\,g cukru
\item 45\,g masla
\item 1/4 balenia kypriaceho prášku
\item 10\,g kakaa
\item 2 žĺtky
\item kvapka (olivového) oleja
\item štipka soli
\end{itemize}
\end{minipage}
\begin{minipage}[t]{0.5\textwidth}
\subsection*{Náplň:}
\begin{itemize}[itemsep=0pt]
\item 150\,g nutelly \\(krému z liekových orieškov)
\item 50\,g horkej čokolády \\(alebo čokolády na varenie)
\item 150\,ml smotany na šľahanie
\item 1 vajce
\end{itemize}
\end{minipage}
\subsection*{Postup}
\setlength{\parskip}{0.5em}
Zo všetkých ingrediencií z prvého zoznamu vymiesime vláčne cesto. Vložíme ho v jednom kuse do sáčku (alebo zabalíme do potravinovej fólie) a odložíme aspoň na 1/2 hodinu do chladničky. 

Vo vodnom kúpeli (miska na hrnci s vodou) rozpustíme nutellu, čokoládu a smotanu a vymiešame do homogénnej zmesi. Odložíme bokom. Po vychladnutí (aspoň čiastočnom) vmiešame celé vajce (poriadne premiešame, aby v plnke nezostali nerozmiešané kusy vajca).

Tortovú formu (okrúhlu, menšiu) vystelieme papierom na pečenie (alebo vymastíme maslom a vysypeme hrubou múkou). Cesto rukami natlačíme do formy. Vytvoríme rovnomernú vrstvu na spodku a párcentimetrový okraj.

Vymiešaný krém vlejeme na cesto, od hornej hranice krému po vrch okrajov cesta by malo zostať aspoň 0.5 cm (náplň ešte trochu narastie). Pečieme asi 20 minút na 180\,${}^\circ$C.

\textit{Poznámka: Kvôli veľkému množstvu prítomných bola na seminári verzia asi z dvojnásobnej dávky. Nugátová platňa z Krabovho jukeboxu je dosť sýta a sladká, takže na bežné posedenie stačí obyčajná dávka.}
\end{document}
